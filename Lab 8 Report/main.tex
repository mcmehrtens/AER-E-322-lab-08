% AER E 361 Mission Report Template
% Spring 2023
% Template created by Yiqi Liang and Professor Matthew Nelson

% Document Configuration DO NOT CHANGE
\documentclass[12 pt]{report}
% --------------------LaTeX Packages---------------------------------
% The following are packages that are used in this report.
% DO NOT CHANGE ANY OF THE FOLLOWING OR YOUR REPORT WILL NOT COMPILE
% -------------------------------------------------------------------

\usepackage{hyperref}
\usepackage{parskip}
\usepackage{titlesec}
\usepackage{titling}
\usepackage{graphicx}
\usepackage{graphviz}
\usepackage[T1]{fontenc}
\usepackage{titlesec, blindtext, color} %for LessIsMore style
\usepackage{tcolorbox} %for references box
\usepackage[hmargin=1in,vmargin=1in]{geometry} % use 1 inch margins
\usepackage{float}
\usepackage{tikz}
\usepackage{svg} % Allows for SVG Vector graphics
\usepackage{textcomp, gensymb} %for degree symbol
\hypersetup{
	colorlinks=true,
	linkcolor=blue,
	urlcolor=cyan,
}
\usepackage{biblatex}
\addbibresource{main.bib}
\usepackage{amsmath}
\usepackage{listings}
\usepackage{multicol}
\usepackage{array}

\usepackage{hologo} %KYR: for \BibTeX
%\usepackage{algpseudocode}
%\usepackage{algorithm}
% This configures items for code listings in the document
\usepackage{xcolor}

\usepackage{fancyhdr} % Headers/Footers
\usepackage{siunitx} % SI units
\usepackage{csquotes} % Display Quote
\usepackage{microtype} % Better line breaks

\definecolor{commentsColor}{rgb}{0.497495, 0.497587, 0.497464}
\definecolor{keywordsColor}{rgb}{0.000000, 0.000000, 0.635294}
\definecolor{stringColor}{rgb}{0.558215, 0.000000, 0.135316}
\definecolor{mygreen}{rgb}{0,0.6,0}
\definecolor{mygray}{rgb}{0.5,0.5,0.5}
\definecolor{mymauve}{rgb}{0.58,0,0.82}

\lstdefinestyle{customc}{
  belowcaptionskip=1\baselineskip,
  breaklines=true,
  frame=L,
  xleftmargin=\parindent,
  language=C,
  showstringspaces=false,
  basicstyle=\footnotesize\ttfamily,
  keywordstyle=\bfseries\color{green!40!black},
  commentstyle=\itshape\color{purple!40!black},
  identifierstyle=\color{blue},
  stringstyle=\color{orange},
 }

 \lstset{ %
  backgroundcolor=\color{white},   % choose the background color; you must add \usepackage{color} or \usepackage{xcolor}
  basicstyle=\footnotesize,        % the size of the fonts that are used for the code
  breakatwhitespace=false,         % sets if automatic breaks should only happen at whitespace
  breaklines=true,                 % sets automatic line breaking
  captionpos=b,                    % sets the caption-position to bottom
  commentstyle=\color{commentsColor}\textit,    % comment style
  deletekeywords={...},            % if you want to delete keywords from the given language
  escapeinside={\%*}{*)},          % if you want to add LaTeX within your code
  extendedchars=true,              % lets you use non-ASCII characters; for 8-bits encodings only, does not work with UTF-8
  frame=tb,	                   	   % adds a frame around the code
  keepspaces=true,                 % keeps spaces in text, useful for keeping indentation of code (possibly needs columns=flexible)
  keywordstyle=\color{keywordsColor}\bfseries,       % keyword style
  language=Python,                 % the language of the code (can be overrided per snippet)
  otherkeywords={*,...},           % if you want to add more keywords to the set
  numbers=left,                    % where to put the line-numbers; possible values are (none, left, right)
  numbersep=8pt,                   % how far the line-numbers are from the code
  numberstyle=\tiny\color{commentsColor}, % the style that is used for the line-numbers
  rulecolor=\color{black},         % if not set, the frame-color may be changed on line-breaks within not-black text (e.g. comments (green here))
  showspaces=false,                % show spaces everywhere adding particular underscores; it overrides 'showstringspaces'
  showstringspaces=false,          % underline spaces within strings only
  showtabs=false,                  % show tabs within strings adding particular underscores
  stepnumber=1,                    % the step between two line-numbers. If it's 1, each line will be numbered
  stringstyle=\color{stringColor}, % string literal style
  tabsize=2,	                   % sets default tabsize to 2 spaces
  title=\lstname,                  % show the filename of files included with \lstinputlisting; also try caption instead of title
  columns=fixed                    % Using fixed column width (for e.g. nice alignment)
}

\lstdefinestyle{customasm}{
  belowcaptionskip=1\baselineskip,
  frame=L,
  xleftmargin=\parindent,
  language=[x86masm]Assembler,
  basicstyle=\footnotesize\ttfamily,
  commentstyle=\itshape\color{purple!40!black},
}

\lstset{escapechar=@,style=customc}

\titlelabel{\thetitle.\quad}

% From here on out you can start editing your document
\newcommand{\subtitle}[1]{%
  \posttitle{%
    \par\end{center}
    \begin{center}\LARGE#1\end{center}
    \vskip0.5em}%
}

\title{\textbf{Iowa State University
\\{\Large Aerospace Engineering}}}
\subtitle{AER E 322 Lab 8\\
		  Thin-Walled Section and Shear Center}
\author{Matthew Mehrtens, Peter Mikolitis, and Natsuki Oda}

\newcommand{\etal}{\textit{et al}., }
\newcommand{\ie}{\textit{i}.\textit{e}., }
\newcommand{\eg}{\textit{e}.\textit{g}., }

% Define the headers and footers
\setlength{\headheight}{70.63135pt}
\geometry{head=70.63135pt, includehead=true, includefoot=true}
\fancypagestyle{plain}{
	\fancyhead{}\fancyfoot{} % clears the headers/footers
	\fancyhead[L]{\textbf{AER E 322}}
	\fancyhead[C]{\textbf{Aerospace Structures Laboratory Report}\\
					 \textbf{Lab 8 Thin-Walled Section and Shear Center}\\
					 Section 4 Group 2\\
					 Matthew Mehrtens, Peter Mikolitis, and Natsuki Oda\\
					 \today}
	\fancyhead[R]{\textbf{Spring 2023}}
	\fancyfoot[C]{\thepage}
}
\pagestyle{fancy}
\fancyhead{}\fancyfoot{} % clears the headers/footers
\fancyhead[L]{\textbf{AER E 322}}
\fancyhead[C]{\textbf{Aerospace Structures Laboratory Report}\\
			  \textbf{Lab 8 Thin-Walled Section and Shear Center}\\
			  Section 4 Group 2\\
			  Matthew Mehrtens, Peter Mikolitis, and Natsuki Oda\\
			  \today}
\fancyhead[R]{\textbf{Spring 2023}}
\fancyfoot[C]{\thepage}

\begin{document}
\maketitle
\tableofcontents

\chapter{Pre-Lab} \label{pre-lab}
\section{Introduction} \label{introduction}
To reduce the weight and cost of aircraft, aerospace engineers use thin-walled structures throughout aerostructures, particularly in the wings. In addition to being more weight and cost effective than traditional structures, they are generally equally strong. Unfortunately, one of the side effects of thin-walled structures is their tendency to twist under unsymmetrical loadings.

To evaluate the strength of thin-walled structures, we have to determine where the shear center is, \ie the point where applied loads bend and do not twist the structure. In this lab, we learned how to calculate the shear center and deflection of a thin-walled structure.

\section{Objectives} \label{objectives}
Using the five provided specimens, we will apply twisting loads to thin-walled beams and make a number of measurements for each specimen. We will use the material learned in lecture and in the lecture notes to calculate the shear center of the thin-walled beams. 

\section{Hypothesis} \label{hypothesis}
Although our individual results from prelab varied significantly from each other, we expect the actual shear center values to at least correlate to the answers we calculated in prelab, even if they are off by a constant or a proportionate. We predict the cross sectional and the angle of opening will have a significant impact on the materials shear center and resistance or tendency to twist.

\chapter{Lab Work} \label{lab_work}
\section{Variables} \label{variables}
\subsection{Independent Variables} \label{variables-independent_variables}
\begin{itemize}
	\item Weight of mass; the applied load varied amongst the specimens
	\item Material of the specimen; the material of the element affects the local stiffness or strength and its tendency to twist
	\item Cross-section of the specimen; this variable controls the value of I which is used to derive the location of the shear center 
\end{itemize}

\subsection{Dependent Variables} \label{variables-dependent_variables}
\begin{itemize}
	\item Deflection due to load; these deflections take the form of bending as observed in cantilevered beams or torsional deflections in the form of twist 
	\item Shear center; the point about which applied loads cause bending but not torsion. The shear center is dependent on the cross-sectional layout. 
\end{itemize}

\section{Work Assignments} \label{work_assignments}
Refer to Table \ref{table:work_assignments} for the distribution of work during this lab.

\begin{table}[!htbp]
\caption{Work assignments for AER E 322 Lab 8.}
\begin{center}
	\begin{tabular}{| c | c | c | c |}
		\hline
		\multicolumn{1}{| c |}{\textbf{Task}} & \textbf{Matthew} & \textbf{Peter} & \textbf{Natsuki} \\
		\hline
		\multicolumn{4}{| c |}{\textit{Lab Work}} \\
		\hline
		Date Recording & X & X & X \\
		\hline
		Exp. Setup & X & X & X \\
		\hline
		Exp. Work & X & X & X \\
		\hline
		Exp. Clean-Up & X & X & X \\
		\hline
		\multicolumn{4}{| c |}{\textit{Report}} \\
		\hline
		Introduction & X & & X \\
		\hline
		Objectives & & & X \\
		\hline
		Hypothesis & X & & X \\
		\hline
		Variables & & X & \\
		\hline
		Materials & & X & \\
		\hline
		Apparatus & & X & \\
		\hline
		Procedures & & X & \\
		\hline
		Data & & & \\
		\hline
		Analysis & X & X & X \\
		\hline
		Conclusion & X & X & \\
		\hline
		References & X & & \\
		\hline
		Appendix & X & & \\
		\hline
		Revisions & X & X & \\
		\hline
		Editing & X & & \\
		\hline
	\end{tabular}
\end{center}
\label{table:work_assignments}
\end{table}

\section{Materials} \label{materials}
\begin{itemize}
	\item Five beam specimens as described in Table \ref{tbl:specimens}
	\item \qty{100}{\g} and \qty{200}{\g} mass
	\item Ruler
	\item Clamp; to secure one end of the beams
	\item Level
\end{itemize}

\begin{table}[!htbp]
\caption{Dimensions and specifications for the five different specimens.}
\begin{center}
	\begin{tabular}{|c|c|c|c|c|c|c|}
		\hline
		Specimen &Cross section &Height, &Width, &Thickness, &Outer &Opening \\
		ID&type&$h$ (\unit{inch})&$b$ (\unit{inch})&$t$ (\unit{inch})&diameter, &angle, \\
		&&&&&$OD$ (\unit{inch})&$2\theta_0$ (\unit{deg})\\
		\hline
		I&Plastic C-channel&2.43&1.456&0.08&N/A&N/A\\
		\hline
		II&Metal C-channel&0.84&0.56&0.055&N/A&N/A\\
		\hline
		III&PVC circular open&N/A&N/A&0.071&1.66&3.1\\
		\hline
		IV&PVC circular open&N/A&N/A&0.071&1.66&36.3\\
		\hline
		V&PVC circular open&N/A&N/A&0.071&1.66&103.7\\
		\hline
	\end{tabular}
\end{center}
\label{tbl:specimens}
\end{table}

\section{Apparatus} \label{apparatus}
The five beams should be cantilevered with the clamp. On the free end, there is a cross beam on which the weight will be hung. Figure \ref{fig:cbar} shows the apparatus for the C-channel beam and Figure \ref{fig:pvcbar} shows the apparatus for the circular PVC beam.

\begin{figure}[htbp]
	\centering
	\includegraphics[width=6in]{images/c-channel beam}
	\caption{One of the cantilevered C-channel beams from lab eight \cite{demo_video}.}
	\label{fig:cbar}
\end{figure}
\begin{figure}[htbp]
	\centering
	\includegraphics[width=6in]{images/circular pvc beam}
	\caption{One of the cantilevered circular PVC beams from lab eight \cite{demo_video}.}
	\label{fig:pvcbar}
\end{figure}

\section{Procedures} \label{procedures}
Set up the five beams as described and shown in Section \ref{apparatus}. Ensure the beams and the table are level. For specimen one, perform a preliminary measurement: measure the length from the edge of the clamp to the cross bar and measure the height from the table to a reference point at the same length as the cross bar, \eg the bottom of the beam.

For each of the five beams, perform the following procedure at least four times---placing the weight at least twice on each side of the cross beam:
\begin{enumerate}
	\item Level the cross bar as best you can. For the circular specimens, this can be done by rotating the beam. Measure the height at both ends of the cross bar to calculate the initial angle of twist. For specimens one and two, if the initial angle of twist is greater than \qty{1}{\degree}, account for this baseline by subtracting the initial angle of twist from subsequent angle of twist calculations.
	\item Place a weight on the cross beam at an arbitrary location (at least twice on both sides). Use the \qty{100}{\g} weight for specimens one, three, and five, and use the \qty{200}{\g} weight for specimens two and four.
	\item Record the location of the weight along the cross bar and the height of both sides of the cross beam. Since the shear center is measured from the center of the circular beams and from the center of the vertical web of the C-beams, make sure to note where your cross beam measurements are referenced from. Later in the analysis you can adjust your cross beam measurements to be relative to the shear center.
\end{enumerate}

For specimen one, put the weight at the shear center, \ie the point where an applied weight causes no twist, and measure the height at the reference point again. This should give you beam deflection.

\section{Data} \label{data}
\textbf{Note:} Our group misinterpreted the lab instructions for this lab. Instead of measuring the height of the cross beam at both ends, we measured the height of the actual beam at both ends. The numbers we calculated in lab are, of course, meaningless. After looking through the lab instruction more closely, it is obvious we were supposed to measure the height of the cross beam to calculate the angle of twist.

To still be able to do the analysis properly, we asked a fellow group in our AER E 322 section, Section \num{4} Group \num{3}, if we could use their data. They agreed; all data used in this report is courtesy of lab group \num{3} in AER E 322 section \num{4}.

To calculate angle of twist, we used Equation \ref{eqn:aot} from the lab eight lecture notes \cite{lecture_notes}:
\begin{align} \label{eqn:aot}
	\theta&=\sin^{-1}\left(\frac{h_R-h_L}{L}\right)
\end{align}
where $h_R$ is the height from the table to the right end of the cross bar, $h_L$ is the height from the table to the left end of the cross bar, and $L$ is the length of the cross bar. Figure \ref{fig:aot} shows this angle visually.

\begin{figure}[htbp]
	\centering
	\includegraphics[width=4in]{images/aot}
	\caption{Visual representation of the angle of twist of the cross bar. \cite{lecture_notes}}
	\label{fig:aot}
\end{figure}

The data we used in the analysis portion of this lab is shown in Table \ref{tbl:data}.

\begin{table}[!htbp]
\caption{Data collected from lab eight, courtesy of lab group \num{3} from AER E 322 section \num{4}. $h$ is the height of the cross bar at the left and right ends, $L$ is the length of the cross bar, $\theta$ is the angle of twist, calculated using Equation \ref{eqn:aot}, $x$ is the distance to the weight on the cross bar, and $m$ is the mass of the weight.}
\begin{center}
	\begin{tabular}{|c|c|c|c|c|c|c|c|}
		\hline
		Specimen&$h_L$ (\unit{\cm})&$h_R$ (\unit{\cm})&$L$ (\unit{\cm})&$\theta$ (\unit{\degree})&$x_L$ (\unit{cm})&$x_R$ (\unit{cm})&$m$ (\unit{\g})\\
		\hline
		\num{1}&\num{10.4}&\num{13.1}&\num{44.0}&\num{3.52}&\num{11.0}&N/A&\num{100}\\
		\hline
		\num{1}&\num{8.8}&\num{14.6}&\num{44.0}&\num{7.57}&\num{14.0}&N/A&\num{100}\\
		\hline
		\num{1}&\num{17.7}&\num{4.8}&\num{44.0}&\num{-17.0}&N/A&\num{9.2}&\num{100}\\
		\hline
		\num{1}&\num{18.7}&\num{3.4}&\num{44.0}&\num{-20.3}&N/A&\num{12.8}&\num{100}\\
		\hline
		\num{2}&\num{9.9}&\num{11.4}&\num{30.0}&\num{2.87}&\num{3.1}&N/A&\num{200}\\
		\hline
		\num{2}&\num{8.1}&\num{13.0}&\num{30.0}&\num{9.40}&\num{9.0}&N/A&\num{200}\\
		\hline
		\num{2}&\num{11.6}&\num{9.5}&\num{30.0}&\num{-4.01}&N/A&\num{3.3}&\num{200}\\
		\hline
		\num{2}&\num{12.6}&\num{8.3}&\num{30.0}&\num{-8.24}&N/A&\num{7.5}&\num{200}\\
		\hline
		\num{3}&\num{7.7}&\num{12.5}&\num{30.0}&\num{9.21}&\num{4.9}&N/A&\num{100}\\
		\hline
		\num{3}&\num{6.9}&\num{13.4}&\num{30.0}&\num{12.5}&\num{7.4}&N/A&\num{100}\\
		\hline
		\num{3}&\num{12.1}&\num{8.2}&\num{30.0}&\num{-7.47}&N/A&\num{9.8}&\num{100}\\
		\hline
		\num{3}&\num{11.5}&\num{9.1}&\num{30.0}&\num{-4.59}&N/A&\num{5.6}&\num{100}\\
		\hline
		\num{4}&\num{9.8}&\num{13.4}&\num{30.0}&\num{6.89}&\num{5.0}&N/A&\num{200}\\
		\hline
		\num{4}&\num{8.9}&\num{14.7}&\num{30.0}&\num{11.1}&\num{7.6}&N/A&\num{200}\\
		\hline
		\num{4}&\num{15.3}&\num{6.1}&\num{30.0}&\num{-17.9}&N/A&\num{7.5}&\num{200}\\
		\hline
		\num{4}&\num{14.2}&\num{7.7}&\num{30.0}&\num{-12.5}&N/A&\num{4.5}&\num{200}\\
		\hline
		\num{5}&\num{10.0}&\num{13.8}&\num{30.5}&\num{7.16}&\num{4.5}&N/A&\num{100}\\
		\hline
		\num{5}&\num{7.6}&\num{16.4}&\num{30.5}&\num{16.8}&\num{10.8}&N/A&\num{100}\\
		\hline
		\num{5}&\num{15.0}&\num{7.2}&\num{30.5}&\num{-14.8}&N/A&\num{5.6}&\num{100}\\
		\hline
		\num{5}&\num{17.2}&\num{5.8}&\num{30.5}&\num{-21.9}&N/A&\num{10.6}&\num{100}\\
		\hline
	\end{tabular}
\end{center}
\label{tbl:data}
\end{table}

\chapter{Conclusion} \label{conclusion-chapter}
\section{Analysis} \label{analysis}
\subsection{Problem 1}
The measured shear centers were calculated with MATLAB by using the \texttt{polyfit} fit. \texttt{polyfit} takes the $x$ and $y$ values as an input---in our case the $x$ and $\theta$ values---as well as an integer specifying what order line of best fit is desired and returns the corresponding polynomial coefficients. In this lab, the shear center data is generally linear, so we use a first order \texttt{polyfit} function. After being called, \texttt{polyfit} returns two values $C_1$ and $C_2$ which form the following linear equation:
\begin{align*}
	y&=C_1x+C_2
\end{align*}
To find the $x$-intercept, we let $y=0$ and solve for $x$. This is shown below:
\begin{align*}
	0&=C_1x+C_2\\
	x&=-\frac{C_2}{C_1}
\end{align*}
This calculated distance is the measured distance from the reference center to the shear center of a material. The scatter plots, lines of best fit, and the point of shear center are shown in the graphs in Appendix \ref{sec:shear_center_graphs}.

The theoretical shear centers were calculated using MATLAB and referring to the formulas from prelab \cite{prelab_instructions}, namely Equations \ref{eqn:sc_cbeam} and \ref{eqn:sc_circbeam}.
\begin{align}
	\text{C-channel: }e&=\frac{h^2b^2t}{4I}\label{eqn:sc_cbeam}\\
	\text{Circular open-channel: }e&=\frac{2r[\cos\theta_0(2\pi-2\theta_0)+2\sin\theta_0]}{2\pi-2\theta_0+\sin(2\theta_0)}\label{eqn:sc_circbeam}
\end{align}
The measured shear centers and the theoretical shear center are shown in Table \ref{tbl:shear_centers}.

\begin{table}[!htbp]
\caption{Dimensions and specifications for the five different specimens.}
\begin{center}
	\begin{tabular}{|c|c|c|c|}
		\hline
		Specimen&$e_\text{meas}$ (\unit{cm})&$e_\text{theor}$  (\unit{cm})&Error (\unit{\percent})\\
		\hline
		1&\num{-0.800}&\num{-1.48}&\num{45.8}\\
		\hline
		2&\num{-0.322}&\num{-0.590}&\num{45.4}\\
		\hline
		3&\num{0.798}&\num{-4.03}&\num{119.8}\\
		\hline
		4&\num{-3.77}&\num{-3.87}&\num{2.79}\\
		\hline
		5&\num{-3.52}&\num{-3.21}&\num{9.54}\\
		\hline
	\end{tabular}
\end{center}
\label{tbl:shear_centers}
\end{table}

Our theoretical calculations matched well for specimens four and five, but poorly for specimens one through three, with the worst match being for specimen three. If the outlier is removed from the specimen three data set, as shown in Figure \ref{fig:specimen_3_sc_modified} in Appendix \ref{sec:shear_center_graphs}, the error drops to a marginally improved \qty{103.1}{\percent}. Possible explanations for the source of this error are explained in Section \ref{sources_of_error}.

\subsection{Problem 2}
First, we calculate the beam deflection theoretically using the method of superposition. Specifically, we refer to the max deflection formula, Equation \ref{eqn:deflection}, for a cantilevered beam shown on slide \num{14} of the week seven lecture notes \cite{week_7_lecture_notes}.
\begin{align}\label{eqn:deflection}
	\nu_\text{max}&=\frac{-PL^3}{3EI}
\end{align}
where $\nu_\text{max}$ is the maximum deflection of the beam, $P$ is the applied load, $L$ is the length of the beam, $E$ is the Young's Modulus, and $I$ is the moment of inertia. Substituting the proper values for this specimen, we find that
\begin{align*}
	\nu_\text{max}&=\frac{-(\qty{0.100}{\kg})(\qty{9.81}{\m\per\s\squared})(\qty{1.10}{\m})^3}{(\qty{3e9}{\Pa})(\qty{1.792e-7}{\m^4})}\\
	&=\qty{-2.43}{\mm}
\end{align*}
In lab, we measured the deflection to be $\qty{15.0}{\cm}-\qty{15.2}{\cm}=\qty{-0.2}{\cm}=\qty{-2}{\mm}$. This very closely matches the theoretical calculation, the limiting factor being the ruler we used to measure the height.

\subsection{Problem 3}
We will use Figure \ref{fig:cross_sections} from the lab eight instructions \cite{lab_instructions} as a visual guide to our shear flow derivations.

\begin{figure}[htbp]
	\centering
	\includegraphics[width=4in]{images/cross sections}
	\caption{Schematic diagrams of the thin-walled cross section of the specimen: (left) C-channel and (right) circular open channel.}
	\label{fig:cross_sections}
\end{figure}

To calculate the shear stress distribution, we use Equation \ref{eqn:shear_stress_distribution} from lecture notes \cite{lecture_notes}
\begin{align} \label{eqn:shear_stress_distribution}
	\tau&=-\frac{P}{It}\int_0^sytds
\end{align}
where $P$ is an external load, $I$ is the moment of inertia, $s$ is the path along the cross section of the beam, and $y$ is the distance from the axis of symmetry. We calculate $I$ as shown in Equation \ref{eqn:I_equation}
\begin{align} \label{eqn:I_equation}
	I&=\frac{2bt^3+t(h-t)^3+6bth^2}{12}
\end{align}
Below is our derivation for the shear stress distribution in each of the three sections of the C-channel beam. Note that since $t$ is small and constant, we can simplify many of the integrals.
\begin{align*}
	\tau_{12}(s_1)&=-\frac{P}{I}\int_0^{s_1}yds_1,\quad\text{where }0\le{}s_1\le{}b\\
	&=-\frac{P}{I}\int_0^{s_1}\left(-\frac{h}{2}\right)ds_1\\
	&=\frac{Phs_1}{2I}
\end{align*}
\begin{align*}
	\tau_{24}(s_2)&=\tau_2-\frac{P}{I}\int_0^{s_2}yds_2,\quad\text{where }0\le{}s_2\le{}h\\
	&=\tau_2-\frac{P}{I}\int_0^{s_2}\left(s_2-\frac{h}{2}\right)ds_2\\
	&=\tau_2-\frac{P}{I}\left(\frac{s_2^2}{2}-\frac{hs_2}{2}\right)\\
	&=\tau_2-\frac{P}{2I}(s_2^2-hs_2)
\end{align*}
\begin{align*}
	\tau_{45}(s_4)&=\tau_4-\frac{P}{I}\int_0^{s_4}yds_4,\quad\text{where }0\le{}s_4\le{}b\\
	&=\tau_4-\frac{P}{I}\int_0^{s_4}\left(\frac{h}{2}\right)ds_4\\
	&=\tau_4-\frac{Phs_4}{2I}
\end{align*}
Programming these equations into MATLAB, for specimen one, we find the shear stress distributions shown below and the values of shear stress notated in Table \ref{tbl:shear_stresses_1}. Both the distributions and shear stresses are shown in Figure \ref{fig:shear_stresses_1}.
\begin{align*}
	\tau_{12}&=s_1(\qty{168.9}{\kilo\pascal\per\meter})\\
	\tau_{24}&=\qty{6.248}{\kilo\pascal}+(\qty{2737}{\kilo\pascal\per\meter\squared})(-s_2^2+s_2(\qty{0.06172}{\meter}))\\
	\tau_{45}&=\qty{6.248}{\kilo\pascal}+s_4(\qty{-168.9}{\kilo\pascal\per\meter})
\end{align*}

\begin{table}[!htbp]
\caption{Shear stresses in specimen one.}
\begin{center}
	\begin{tabular}{|c|c|}
		\hline
		Point&$\tau$ (\unit{\kilo\pascal})\\
		\hline
		\num{1}&\num{0}\\
		\hline
		\num{2}&\num{6.248}\\
		\hline
		\num{3}&\num{8.854}\\
		\hline
		\num{4}&\num{6.248}\\
		\hline
		\num{5}&\num{0}\\
		\hline
	\end{tabular}
\end{center}
\label{tbl:shear_stresses_1}
\end{table}

\begin{figure}[htbp]
	\centering
	\includegraphics[width=5in]{images/shear_stresses_1}
	\caption{Shear stresses and shear stress distribution of specimen one.}
	\label{fig:shear_stresses_1}
\end{figure}

For specimen two, we find the shear stress distributions shown below and the values of shear stress notated in Table \ref{tbl:shear_stresses_2}. Both the distributions and shear stresses are shown in Figure \ref{fig:shear_stresses_2}.
\begin{align*}
	\tau_{12}&=s_1(\qty{3839}{\kilo\pascal\per\meter})\\
	\tau_{24}&=\qty{54.61}{\kilo\pascal}+(\qty{1.799e5}{\kilo\pascal\per\meter\squared})(-s_2^2+s_2(\qty{0.02134}{\meter}))\\
	\tau_{45}&=\qty{54.61}{\kilo\pascal}+s_4(\qty{-3839}{\kilo\pascal\per\meter})
\end{align*}

\begin{table}[!htbp]
\caption{Shear stresses in specimen two.}
\begin{center}
	\begin{tabular}{|c|c|}
		\hline
		Point&$\tau$ (\unit{\kilo\pascal})\\
		\hline
		\num{1}&\num{0}\\
		\hline
		\num{2}&\num{54.61}\\
		\hline
		\num{3}&\num{75.08}\\
		\hline
		\num{4}&\num{54.61}\\
		\hline
		\num{5}&\num{0}\\
		\hline
	\end{tabular}
\end{center}
\label{tbl:shear_stresses_2}
\end{table}

\begin{figure}[htbp]
	\centering
	\includegraphics[width=5in]{images/shear_stresses_2}
	\caption{Shear stresses and shear stress distribution of specimen two.}
	\label{fig:shear_stresses_2}
\end{figure}

\subsection{Problem 4}\label{problem_4}
Since the C-channel beam is symmetrical about the $x$-axis, the shear center must lie on the $x$-axis. If it was above or below the $x$-axis, force applied at that location would cause torsion. In both specimen, the $x$-axis is located at $\frac{h}{2}$. The $x$-coordinate for the shear centers is found using Equation \ref{eqn:sc_cbeam} and is tabulated in Table \ref{tbl:shear_centers}.

Assuming the $x$ axis aligns with the axis of symmetry for the C-channel beam and the $y$ axis aligns with the axis of symmetry for the vertical web of the web, the exact $(x,y)$ coordinate positions for specimen one and two are as follows.
\begin{align*}
	\text{Specimen 1:}\quad(x,y)&=(\qty{-1.48}{\cm},\num{0})\\
	\text{Specimen 2:}\quad(x,y)&=(\qty{-0.590}{\cm},\num{0})
\end{align*}
These coordinates are shown visually with a red dot in Figures \ref{fig:sc_grid_1} and \ref{fig:sc_grid_2}. Each square of the grid is equivalent to a square with dimensions $t\times{}t$, where $t$ is the thickness of the beam.

\begin{figure}[htbp]
	\centering
	\includegraphics[width=4in]{images/shear_center_1}
	\caption{Shear center grid for specimen one. The shear center is marked with the red dot. Each grid square is a $t\times{}t$ square, where $t$ is the thickness of the beam.}
	\label{fig:sc_grid_1}
\end{figure}

\begin{figure}[htbp]
	\centering
	\includegraphics[width=4in]{images/shear_center_2}
	\caption{Shear center grid for specimen two. The shear center is marked with the red dot. Each grid square is a $t\times{}t$ square, where $t$ is the thickness of the beam.}
	\label{fig:sc_grid_2}
\end{figure}

\subsection{Problem 5}
If the height of the C-channel beam were to increase a hundredfold or approach infinity, the shape of the C-channel beam would begin to resemble a rectangle. Specifically, the horizontal flanges would have less of an effect on the structure. As the vertical flange increases in area, the shear center will move closer to the vertical axis of symmetry of the vertical flange. Since $e$ is measured from that vertical axis of symmetry, $e$ should approach \num{0}. We can prove this by substituting $I$ from Equation \ref{eqn:I_equation} into Equation \ref{eqn:sc_cbeam} and taking the limit as $h$ approaches infinity. This proof is shown below:
\begin{align*}
	e&=\frac{h^2b^2t}{4I}\\
	&=\frac{12h^2b^2t}{2bt^3+t(h-t)^3+6bth^2}\\
	&=\frac{12h^2b^2t}{2bt^3+th^3-3t^2h^2+3t^3h-t^4+6bth^2}\\
	&=\frac{12h^2b^2}{2bt^2+h^3-3th^2+3t^2h-t^3+6bh^2}
\end{align*}
Since there is an $h^3$ term in the denominator and only an $h^2$ term in the numerator, the denominator will grow faster than the numerator and therefore,
\begin{align*}
	\lim_{h\rightarrow\infty}(e)&=\lim_{h\rightarrow\infty}\left(\frac{12h^2b^2}{2bt^2+h^3-3th^2+3t^2h-t^3+6bh^2}\right)\\
	&=0
\end{align*}

\subsection{Problem 6}
During the increase of both opening angle and radius, shape of the section going not to look like $C$. Increasing open angle, the open portion gets itself bigger. If the open angle reaches \qty{180}{\degree} ($2\theta_0=\qty{180}{\degree}$), shape of cross section will look like the circular pipe split in half along to path of that. However, the area value of this section will be bigger, depending on the value of increment. The equation below expresses the area of section with the circular given:
\begin{align*}
	A&=\pi\left(r+\frac{t}{2}\right)^2\frac{360-2\theta_0}{360}-\pi\left(r-\frac{t}{2}\right)^2\frac{360-2\theta_0}{360}
\end{align*}
The area will increase depending on how the value of the open-angle and radius increase. That is because the radius increases quadratically. But after open angle reaches \qty{90}{\degree}, the value will start to decrease because shear center will approach radius.

\begin{figure}[htbp]
	\centering
	\includegraphics[width=4in]{images/Graphs/random_graph}
	\caption{The changes when both radius and degree of open angle increases one by one.}
	\label{fig:random_graph}
\end{figure}

Plotting Equation \ref{eqn:sc_circbeam}, the equation for the shear center offset of a circular open-channel beam, and letting $\theta_0$ get arbitrarily close to $\pi$, we find that the shear center offset approaches $r$ as $\theta_0\rightarrow\pi$.

\section{Sources of Error}\label{sources_of_error}
Some of our shear center calculations had a significant amount of error. If the lab table or the beam was not initially level, all our measurements would be systematically incorrect. If the thickness of the beams or the dimensions were not exactly as described in the lab instructions, this would also result in higher error. Since we were measuring by hand, there is a possibly for random human error to propagate through our calculations due to the inconsistent way in which we measure things visually. Additionally, we did not ``zero'' out the cross bar after each test, so there may have been an initial angle of twist value after the first test was measured.

\section{Conclusion} \label{conclusion-section}
Due to the numerous sources of error described in Section \ref{sources_of_error}, our shear center calculations did not match well on specimens one through three. We were able to accurately predict the location of the shear center in specimens four and five, however.

Additionally, we accurately predicted the deflection of specimen one undergoing a loading at its shear center. We examined what effect changing the cross section would have on the shear center offset of a C-channel and open-channel beam.

\printbibliography[heading=subbibintoc]
\appendix
\chapter{Graphs}
\section{Shear Center Graphs} \label{sec:shear_center_graphs}
\begin{figure}[htbp]
	\centering
	\includesvg[width=4in,inkscapelatex=false]{images/Graphs/specimen_1_graph}
	\caption{The measured angles of twist for specimen one with the measured shear center.}
	\label{fig:specimen_1_sc}
\end{figure}
\begin{figure}[htbp]
	\centering
	\includesvg[width=4in,inkscapelatex=false]{images/Graphs/specimen_2_graph}
	\caption{The measured angles of twist for specimen two with the measured shear center.}
	\label{fig:specimen_2_sc}
\end{figure}
\begin{figure}[htbp]
	\centering
	\includesvg[width=4in,inkscapelatex=false]{images/Graphs/specimen_3_graph}
	\caption{The measured angles of twist for specimen three with the measured shear center.}
	\label{fig:specimen_3_sc}
\end{figure}
\begin{figure}[htbp]
	\centering
	\includesvg[width=4in,inkscapelatex=false]{images/Graphs/specimen_4_graph}
	\caption{The measured angles of twist for specimen four with the measured shear center.}
	\label{fig:specimen_4_sc}
\end{figure}
\begin{figure}[htbp]
	\centering
	\includesvg[width=4in,inkscapelatex=false]{images/Graphs/specimen_5_graph}
	\caption{The measured angles of twist for specimen five with the measured shear center.}
	\label{fig:specimen_5_sc}
\end{figure}
\begin{figure}[htbp]
	\centering
	\includesvg[width=4in,inkscapelatex=false]{images/Graphs/specimen_3_graph_no_outlier}
	\caption{The measured angles of twist for specimen three with the measured shear center and the final outlier data point removed.}
	\label{fig:specimen_3_sc_modified}
\end{figure}

\chapter{Code}
See the attached pages for our calculation script and output.
\end{document}
